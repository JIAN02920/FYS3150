\section{The Ising model}

We first start by discussing the Ising Model in its most general form. We later
specialize the model to our specific needs.

\subsection{The general Ising model}
The \emph{Ising Model} is a model in statistical physics that describe phase
transitions in ferromagnetism\footnote{Ferromagnetism is the basic mechanism by
which certain materials form permanent magnets, or are attracted to magnets.
Ferromagnetism is the strongest type of magnetism.}. In this model we represent
magnetic dipole moments of atomic spins that can be in one of two states,
$\uparrow$ or $\downarrow$. The spins are arranged in a \emph{lattice}, which
enables us to relate one spin to its neighbors.

We first introduce some notation. Let $\Lambda$ be a set of lattice positions,
where each element in $\Lambda$ is assigned a set of neighboring sites. The
lattice then forms a graph. We assign for each $k \in \Lambda$ a variable
$\sigma_k$ which takes the value 1 or -1, representing the site's spin,
$\uparrow$ or $\downarrow$, respectively. We can now create a \emph{spin
configuration} $\sigma$ which is a specific configuration of this lattice.  For
two sites $i, j \in \Lambda$ one can talk about the \emph{interaction}
$J_{ij}$. Generally, for each $i \in \Lambda$ one also has an \textit{external
magnetic field} $h_i$ interacting with $i$.

We can talk of the \emph{energy} of a specific configuration, which is given by
the Hamiltonian
\begin{equation}
  \label{eq:general ising model}
  E = H(\sigma) = - \sum^{}_{\langle i k\rangle} J_{ij}\sigma_i\sigma_j - \mu \sum^{}_{j} h_j\sigma_j
\end{equation}
where the notation $\langle i j \rangle$ just means that we sum over adjacent
spins only.

In order to calculate relevant physical quantities for such a system, we need
an appropriate probability that describe the probability of finding the system
in a certain configuration. The \emph{configuration probability} is given by
the Boltzmann distribution
\begin{equation}
  \label{eq:configuration probability}
  P_\beta(\sigma) = \frac{e^{-\beta H(\sigma)}}{Z_\beta}
\end{equation}
where $Z_\beta$ is a normalization constant. We call $Z_\beta$ the
\emph{partition function} for the \emph{canonical ensemble}. It is interesting
to note that when increasing the temperature $T$, the probability
$P_\beta(\sigma)$ of finding the system in a specific configuration $\sigma$
\emph{decreases}. This is essentially due to the fact that when we increase the
temperature the probability of finding states in an unfavorable configuration
become probabilistically feasible in comparison with the most common
configurations.

When modeling a magnetic physical system there are certain physical quantities
that we are interested in. Firstly, the \emph{mean energy} given by
\begin{equation}
  \label{eq:mean energy}
  \langle E \rangle = \sum^{M}_{i=1} E_i P_\beta(\sigma) = \frac{1}{Z} \sum^{M}_{i=1} E_i e^{-\beta E_i}.
\end{equation}
The \emph{mean magnetization} is given by
\begin{equation}
  \label{eq:mean magnetization}
  \langle \mathcal{M} \rangle = \sum^{M}_{i=1} \mathcal{M}_i P_\beta(\sigma) =\frac{1}{Z} \sum^{M}_{i=1} \mathcal{M}_i e^{\beta E_i},
\end{equation}
where $\mathcal{M}_i = \sum^{}_{j \in \Lambda}\sigma_j$ for all configurations
$\sigma$ and $M$ denotes the number of possible configurations.  We are also
interested in the \emph{magnetic susceptibility} $\chi$ which tells us how much
an extensive parameter changes when an intensive parameter increases. It is
given by
\begin{equation}
  \label{eq:magnetic susceptibility}
  \chi = \frac{1}{k_BT}\left( \langle \mathcal{M}^2 \rangle - \langle \mathcal{M} \rangle^2 \right).
\end{equation}
The \emph{specific heat} $C_V$, which tells us how much the energy changes by
increasing the temperature, is given by
\begin{equation}
  \label{eq:specific heat}
  C_V = \frac{1}{k_BT^2} \left( \langle E^2 \rangle - \langle E \rangle ^2 \right).
\end{equation}

If we can make some assumptions about the interaction $J_{ij}$ we can classify
the Ising model.  If for all $i, j \in \Lambda$ we have
\begin{enumerate}[i)]
  \item $J_{ij} > 0$, we call the interaction \emph{ferromagnetic};
  \item $J_{ij} < 0$, we call the interaction \emph{antiferromagnetic};
  \item $J_{ij} = 0$, the spins are non-interacting.
\end{enumerate}

\subsection{Ising model specialized to our project}

We can make some simplifications to the general Ising model in order to tailor
it to our specific needs. First of all, we wish to examine the system with no
external magnetic field\footnote{This specific system was solved analytically
by Lars Onsager in 1944.}. We can therefore, due to $h_j = 0$ for all $j \in
\Lambda$, rewrite
\cref{eq:general ising model} as
\begin{equation}
  \label{eq:no magnetic ising}
  E = H(\sigma) = - \sum^{}_{\langle i k \rangle} J_{ij}\sigma_i\sigma_j.
\end{equation}
We also assume that the coupling constant $J_{ij}$ that describe the
interaction between neighboring spins as constant $J$ for all $i, j \in
\Lambda$. \Cref{eq:no magnetic ising} can then be read as
\begin{equation}
  \label{eq:constant interaction ising}
  E = H(\sigma) = - J\sum^{}_{\langle i k \rangle} \sigma_i\sigma_j
\end{equation}

In this project we wish to examine a ferromagnetic system, $J > 0$. This in
turn, means that it is favorable for neighboring spins to be parallel, as this
leads to a lower energy. This can be seen from the fact that $\sigma_i \sigma_j
= 1$ whenever spin $i$ and $j$ have the same sign.

\subsection{A $2\times2$ Ising model}
\label{sub:a_2times2_ising_model}

In order to get a feel for what these systems look like we examine a two
dimensional Ising model with lattice dimensions $L = 2$ and periodic boundary
conditions. Our model has $M = 2^4$ different configurations $\sigma$. The
energy for any arbitrary configuration is given by
\begin{align*}
  \label{eq:abitrary energy}
  E_i = -J \sum^{4}\nolimits_{\langle kl \rangle} \sigma_k\sigma_l.
\end{align*}

Based on this expression we see that whenever all spins are parallel we have a
configuration energy of $E_i = -8J$. In the cases where one single spin is
anti-parallel to the other three we have a configuration energy of $E_i = 0$.
If we have two parallel and two anti-parallel, the configuration energy is $E_i
= 0$ or $E_i = 8J$. Each energy state $E_i$ has a \emph{degeneracy}
$\Omega(E_i)$ which corresponds to the number of configurations with the same
energy state. The ground state for this system has degeneracy 2, so we can
either start with all spin up or all spin down and still be in the lowest
energy state possible. For a small $2\times2$ system such as this one, it is
not unlikely that we can transition from a configuration with all spin up to a
configuration with all spin down (although keeping the same energy level),
however this is unlikely for large lattices.

We are now interested in examining the physical quantities of interest
introduced above.  For the $2\times2$ Ising model these quantities have closed
form expressions. The work needed to show these relations are given in Appendix
A. The partition function for this particular system is given by
\begin{equation}
  \notag
  Z = \sum^{16}_{i=1} e^{-\beta E_i} = 4 \cosh (8\beta J) + 12.
\end{equation}
We can now compute the mean energy of the system. This gives
\begin{equation}
  \notag
  \langle E \rangle = 8J \frac{\sinh(8\beta J)}{\cosh (8\beta J)+3}
\end{equation}
Differentiating once more gives us an expression for the specific heat $C_V$
\begin{equation}
  \notag
  C_V = -\frac{64J^2 }{k_BT^2(\cosh(8\beta J) + 3)} \left( \cosh(8\beta J) - \frac{\sinh^2(8\beta J)}{\cosh(8\beta J)+3} \right)
\end{equation}
Similarly, we have a closed form expression for the susceptibility:
\begin{equation}
  \notag
  \chi = \frac{8 \left( e^{8\beta J} + 1 \right)}{\cosh(8\beta J) + 3} \frac{1}{k_BT}.
\end{equation}
We will later refer back to these specific values for the $2\times2$ Ising
model when we compare the analytical results to the numerically computed
values.
