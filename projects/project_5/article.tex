\documentclass[a4paper, 11pt]{article}

\usepackage[]{amsmath, amsthm, amssymb} 
\usepackage[]{algorithmic} 

\title{$N$-body simulation of an open galactic cluster}
\author{Ivar Haugal{\o}kken Stangeby}

\begin{document}
\maketitle
\begin{abstract}
    In this project we look at the Newtonian $N$-body problem using various
    numerical methods for solving ordinary differential equations. Specifically
    of interest is the fourt order Runge-Kutta method and the Velocity-Verlet
    method.
\end{abstract}

\tableofcontents

\section{$N$-body problem}
\label{sec:_n_body_problem}

In this section we look at the general form of the $N$-body problem, as well as
examine the special case of $N = 2$, the two-body problem.

\subsection{General form}
\label{sub:general_form}

The $N$-body problem considers $N$ point masses denoted $m_i$ with $i = 1, 2,
\ldots N$ in an inertial reference frame in three dimensional space
$\mathbb{R}^3$. To each point mass $m_i$ we can associate a position vector
$\mathbf{q}_i$.  According to newtons second law we have that the sum of the
forces on the mass is equal to the mass times the acceperation. We have from
Newton's law of gravity that the gravitational force felt on $m_i$ by a single
mass $m_j$ ($j \neq i$) is given by
\begin{equation}
    \notag
    \mathbf{F}_{ij} = \frac{Gm_im_j \left( \mathbf{q}_j - \mathbf{q}_i \right)}{\|\mathbf{q}_j - \mathbf{q}_i\|_{}^3}
\end{equation}
with $G$ being the gravitational constant and $\|\mathbf{q}_j -
\mathbf{q}_i\|_{}$ is the magnitude of the distance between $\mathbf{q}_i$ and
$\mathbf{q}_j$.
We can now sum over all masses, which gives us the $N$-body \emph{equations of
motion}:
\begin{equation}
    \notag
    m_i \frac{d^2\mathbf{q}_i}{dt^2} = \sum^{N}_{j=1, j\neq i} \frac{Gm_im_j \left( \mathbf{q}_j \right) - \mathbf{q}_i)}{\|\mathbf{q}_j - \mathbf{q}_i\|_{}^3} = \frac{\partial U}{\partial \mathbf{q}_i}
\end{equation}
where $U$ is the \emph{self-potential} energy
\begin{equation}
    \notag
    U = \sum^{}_{1 \leq i < j \leq N} \frac{Gm_im_j}{\|\mathbf{q}_j-\mathbf{q}_i \|_{}}.
\end{equation}
We can now define the momentum associated to each mass $m_i$ to be $\mathbf{p}_i = m_id\mathbf{q}_i/d_t$, which gives us that \emph{Hamilton's equations of motion} for the $N$-body problem become
\begin{align*}
    \notag
    \frac{d\mathbf{q}_i}{dt} = \frac{\partial H}{\partial \mathbf{p}_i} && \frac{d\mathbf{p}_i}{dt} = -\frac{\partial H}{\partial \mathbf{q}_i},
\end{align*}
where $H = T + U$ and $T$ is the kinetic energy given by
\begin{equation}
    \notag
    T = \sum^{N}_{i=1} \frac{\|\mathbf{p}_i\|_{}^2}{2m_i}.
\end{equation}

\subsection{Two-body problem}
\label{sub:two_body_problem}

If we consider the motion of two bodies, for instance the Earth-Sun system, then:
\begin{align*}
    \mathbf{F}_{12} = \frac{Gm_1m_2 \left( \mathbf{q}_2 - \mathbf{q}_1 \right)}{\|\mathbf{q}_j - \mathbf{q}_i\|_{}^3} && \text{Earth Sun} \\
    \mathbf{F}_{21} = \frac{Gm_2m_1 \left( \mathbf{q}_1 - \mathbf{q}_2 \right)}{\|\mathbf{q}_1 - \mathbf{q}_2\|_{}^3} && \text{Sun Earth}
\end{align*}

\section{Numerical Methods}
\label{sec:numerical_methods}

We examine the numerical methods used in this project, and look at various
properties related to the performance of these methods in $N$-body simulations.
We benchmark using a two-body problem.

\subsection{4-th order Runge-Kutta}
\label{sub:4_th_order_runge_kutta}

In the fourth-order Runge-Kutta method, also known as \emph{the classical
Runge-Kutta method} we determine the next values the approximations using the
value before as well as a weighted average of four increments where each
increment is a multiple of the interval and an estimated slope, given by an
explicit formula for the first derivative.

\subsubsection{Algorithm}
\label{ssub:algorithm}

Specify an initial value problem as
\begin{equation}
    \notag
    \dot{y} = f(t, y), \quad y(t_0) = y_0
\end{equation}
where $f$, $t_0$ and $y_0$ is given. Then the fourth order Runge-Kutta method
can be described as follows:
\begin{algorithmic}
    \FOR{$n = 0$ to $N-1$}
    \STATE $y_{n+1} = y_n + \frac{h}{6}\left( k_1 + 2k_2 + 2k_3 + k_4 \right)$
    \STATE $t_{n+1} = t_n + h$
    \ENDFOR
\end{algorithmic}
where 
\begin{flalign*}
    k_1 &= f(t_n, y_n), & k_2 &= f(t_n + h/2, y_n + k_1 h/2), \\
    k_3 &= f(t_n + h/2, y_n + k_2h/2), &  k_4 &= f(t_n + h, y_n + hk_3).
\end{flalign*}
\subsection{Velocity-Verlet}
\label{sub:velocity_verlet}
\end{document}
