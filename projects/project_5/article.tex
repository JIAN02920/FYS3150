\documentclass[a4paper, 11pt]{article}

\usepackage[]{amsmath, amsthm, amssymb} 

\title{$N$-body simulation of an open galactic cluster}
\begin{document}
\maketitle
\begin{abstract}
    In this project we look at the Newtonian $N$-body problem using various
    numerical methods for solving ordinary differential equations. Specifically
    of interest is the fourt order Runge-Kutta method and the Velocity-Verlet
    method.
\end{abstract}

\section{$N$-body problem}
\label{sec:_n_body_problem}

The $N$-body problem considers $N$ point masses denoted $m_i$ with $i = 1, 2,
\ldots N$ in an inertial reference frame in three dimensional space
$\mathbb{R}^3$. To each point mass $m_i$ we can associate a position vector
$\mathbf{q}_i$.  According to newtons second law we have that the sum of the
forces on the mass is equal to the mass times the acceperation. We have from
Newton's law of gravity that the gravitational force felt on $m_i$ by a single
mass $m_j$ ($j \neq i$) is given by
\begin{equation}
    \notag
    \mathbf{F}_{ij} = \frac{Gm_im_j \left( \mathbf{q}_j - \mathbf{q}_i \right)}{\|g_j - q_i\|_{}^3}
\end{equation}
with $G$ being the gravitational constant and $\|\mathbf{q}_j -
\mathbf{q}_i\|_{}$ is the magnitude of the distance between $\mathbf{q}_i$ and
$\mathbf{q}_j$.
We can now sum over all masses, which gives us the $N$-body \emph{equations of
motion}:
\begin{equation}
    \notag
    m_i \frac{d^2\mathbf{q}_i}{dt^2} = \sum^{N}_{j=1, j\neq i} \frac{Gm_im_j \left( \mathbf{q}_j \right) - \mathbf{q}_i)}{\|\mathbf{q}_j - \mathbf{q}_i\|_{}^3} = \frac{\partial U}{\partial \mathbf{q}_i}
\end{equation}
where $U$ is the \emph{self-potential} energy
\begin{equation}
    \notag
    U = \sum^{}_{1 \leq i < j \leq N} \frac{Gm_im_j}{\|\mathbf{q}_j-\mathbf{q}_i \|_{}}.
\end{equation}
We can now define the momentum associated to each mass $m_i$ to be $\mathbf{p}_i = m_id\mathbf{q}_i/d_t$, which gives us that \emph{Hamilton's equations of motion} for the $N$-body problem become
\begin{align*}
    \notag
    \frac{d\mathbf{q}_i}{dt} = \frac{\partial H}{\partial \mathbf{p}_i} && \frac{d\mathbf{p}_i}{dt} = -\frac{\partial H}{\partial \mathbf{q}_i},
\end{align*}
where $H = T + U$ and $T$ is the kinetic energy given by
\begin{equation}
    \notag
    T = \sum^{N}_{i=1} \frac{\|\mathbf{p}_i\|_{}^2}{2m_i}.
\end{equation}
\end{document}
